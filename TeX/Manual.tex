% Created 2013-06-17 Mon 00:35
\documentclass[11pt]{article}
\usepackage[utf8]{inputenc}
\usepackage[latin1]{inputenc}
\usepackage{euler}    
\usepackage[T1]{fontenc}
\usepackage{fixltx2e}
\usepackage{graphicx}
\usepackage{longtable}
\usepackage{float}
\usepackage{wrapfig}
\usepackage{soul}
\usepackage{textcomp}
\usepackage{marvosym}
\usepackage{wasysym}
\usepackage{latexsym}
\usepackage{amssymb}
\usepackage{hyperref}
\tolerance=1000
\providecommand{\alert}[1]{\textbf{#1}}

\title{Casamentos Estáveis - Algoritmos em Python}
\author{Luciano Monteiro de Castro}
\date{\today}
\hypersetup{
  pdfkeywords={},
  pdfsubject={},
  pdfcreator={Emacs Org-mode version 7.8.11}}

\begin{document}

\maketitle

\setcounter{tocdepth}{3}
\tableofcontents
\vspace*{1cm}

\section{Resumo}
\label{sec-1}

Este documento contém instruções para utilização dos algoritmos contidos no
arquivo \verb|casamentosestaveis.py|, encapsulados no formato de funções. 
O arquivo contém descrições das funções em si, de forma que nosso enfoque aqui
será o de adotar o ponto de vista dos possíveis usuários destes algoritmos,
de acordo com a natureza de seu interesse no problema dos casamentos estáveis.
\section{O Problema dos Casamentos Estáveis}
\label{sec-2}


<Descrição do problema - copiar da wikipedia>
\section{Determinação de um emparelhamento (casamento) estável}
\label{sec-3}

Se você deseja utilizar este pacote para resolver uma instância concreta do problema
dos casamentos estáveis, mesmo que não tenha qualquer experiência em Python, 
pode fazê-lo através das funções \verb|casar| ou 
\verb|casarp|, dependendo do formato em que preferir obter a solução.
Para isso, siga os seguintes passos:
\subsection{Crie listas numeradas de preferências para ambos os grupos}
\label{sec-3-1}

Para facilitar, manteremos a convenção usual e chamaremos os membros de um grupo de
``homens'' e os do outro grupo de ``mulheres''. Você deve numerar tanto os homens como
as mulheres utilizando os números naturais de $0$ a $n-1$, onde $n$ representa a
quantidade de pessoas em cada grupo. Depois, cada homem deve listar as mulheres em
sua ordem de preferência, começando por sua preferida, e cada mulher deve fazer o
mesmo para os homens. As mulheres e os homens nas listas devem ser representados pelos
seus números. Dada a natureza do algoritmo usado na solução, o emparelhamento
encontrado será o melhor possível para os homens, então você pode querer escolher
como grupo dos homens o que mais lhe convier.

{\bf Exemplo:} 

Homens: 0. André, 1. Bruno, 2. Carlos. 

Mulheres: 0. Ana, 1. Bianca, 2. Claudia. 

Se a ordem de preferências de André é Bianca, Claudia, Ana, então a sua
lista será 1, 2, 0. Se a lista de Ana é 2, 1, 0, sua ordem de preferências é Carlos,
Bruno, André.

\subsection{Crie um arquivo de texto que traduza as listas de preferências}
\label{sec-3-2}

Esta etapa requer absoluta precisão. O arquivo de texto deve conter as $2n$ listas de
preferências, começando pelos homens e depois pelas mulheres. Os números em uma mesma
lista devem ser separados por espaços, e uma lista deve ser separada da outra por
ponto e vírgula (;). O arquivo deve constituir-se de uma única linha de texto.

{\bf Exemplo:} Com os homens e mulheres do exemplo anterior, a única linha
de nosso arquivo de texto deve começar com a lista de André: \verb|1 2 0|. Deve
seguir-se um ponto e vírgula e a lista de Bruno, e assim por diante. Ao final o
conteúdo do arquivo será algo assim:

\verb|1 2 0; 0 1 2; 1 0 2; 2 1 0; 0 1 2; 0 2 1|

Observe que as $6$ listas de preferências são facilmente identificáveis, e estão
na ordem André, Bruno, Carlos, Ana, Bianca, Claudia. Há um arquivo como este no
repositório com o nome \verb|exemplo.cas|.
\subsection{Incicie uma sessão de Python}
\label{sec-3-3}

Para isso você precisa instalar uma versão do Python em seu computador. Ver, 
por exemplo, \verb|www.python.org|.

\subsection{Executar a função casarp}
\label{sec-3-4}

Esta função imprime diretamente na tela a lista dos pares de um emparelhamento
estável, cada par contendo primeiro o número do homem, e depois o número da mulher.

Para executá-la, digite primeiro

\verb|>>> import casamentosestaveis as cas|

seguido da tecla `enter'.

(você precisa baixar o arquivo \verb|casamentosestaveis.py| contido neste
repositório, e gravá-lo em uma pasta acessível ao Python, por exemplo raiz ou home).

Depois execute a função \verb|casarp| sobre o arquivo contendo as listas de 
preferências. Para isso digite

\verb|>>> cas.casarp(`nomedoarquivo')|

seguido de `enter'. Substitua ``nomedoarquivo'' pelo nome correto do arquivo, mas 
não esqueça de fazê-lo entre \verb|'| e \verb|'|.

{\bf Exemplo:} Continuando com o mesmo exemplo dos itens anteriores, eis o que
executamos e obtemos:

\begin{verbatim}
>>> import casamentosestaveis as cas
import casamentosestaveis as cas
>>> cas.casarp('exemplo.cas')
cas.casarp('exemplo.cas')
[0, 1]
[1, 2]
[2, 0]
>>> 
\end{verbatim}

Ou seja, o homem $0$, André, fica emparelhado com a mulher $1$, Bianca. Bruno fica
com Claudia e Carlos, com Ana.

\end{document}